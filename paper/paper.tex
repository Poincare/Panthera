\documentclass{article}

% The geometry package allows for easy page formatting.
\usepackage{geometry}
\geometry{letterpaper}

% Load up special logo commands.
\usepackage{doc}

% Package for formatting URLs.
\usepackage{url}

% Packages and definitions for graphics files.
\usepackage{graphicx}
\usepackage{epstopdf}
\usepackage{natbib}
\usepackage{tikz}
\DeclareGraphicsRule{.tif}{png}{.png}{`convert #1 `dirname #1`/`basename #1 .tif`.png}

%
% Set the title, author, and date.
%
\title{Panthera: A Study of Caching in Distributed Computing}
\author{Dhaivat Pandya}
\date{}

%
% The document proper.
%
\begin{document}

% Add the title section.
\maketitle

% Add an abstract.
\abstract{
Describe your paper in 100-200 words, give or take.  The command-line \texttt{wc} utility is really useful here!  This particular sample paper is meant to demonstrate a variety of \LaTeX\ directives for producing a well-structured, consistently-formatted scholarly document.  The actual content and outline may vary according to the needs of your specific research topic.
}

\section{Introduction}
The Hadoop distributed system \cite{hadoop} is an open source version of the revolutionary MapReduce system developed at Google. With it, developers can take easily take advantage of large, multi-node clusters to solve computational problems.

Latency in distributed systems can have significant effects, and a reduction of the same comes with tremendous benefits. As the RAMCloud project has outlined \cite{ramcloud}, low latency can greatly extend the applications of distributed computing systems. 

In this paper, we discuss \textit{Panthera}, a cache layer for Hadoop.

\section{Constraints}

There are several systems built on Hadoop that are in widespread use \cite{hbase, cloudbatch, pig}. Thus, for \textit{Panthera} to be practical, it must operate independently of the existing Hadoop codebase. Additionally, for non-cache related requests, \textit{Panthera} must add insignificant latency. Finally, data and metadata request latency should be significantly with \textit{Panthera} in comparison to a vanilla Hadoop installation.

\section{Architecture}
\begin{figure}
\begin{center}
	\begin{tikzpicture}
		\begin{tikzpicture}[scale=1.5, transform shape]
		\draw[style=solid] (0,0) circle (1) node {DataNode 1};
		\draw[style=dashed] (0, 0) circle(1.3) node {DataNode 1};
		\draw[style=solid] (3.0, 2.5) node {NameNode}  circle (1);
		\draw[style=solid] (3.0, 0) node{DataNode 2} circle (1);
		\draw[style=solid] (6.0, 0) node{DataNode 3} circle (1);
		\draw[loosely dashed, line width=2pt]  (2.5,0.3)-- (2.5, 2.3);
		\draw[loosely dashed, line width=2pt]  (0, 0.3) -- (2.5, 2.3);
		\draw[loosely dashed, line width=2pt]  (5.0, 0.3) -- (2.5, 2.3);
	\end{tikzpicture}
	\end{tikzpicture}
\end{center}
\end{figure}

\textit{Panthera} runs as a standalone program on every slave node (usually also a DataNode) of a given Hadoop Distributed File System cluster. 


\bibliographystyle{plain}

\bibliography{references}

\end{document}